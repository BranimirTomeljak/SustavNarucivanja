\chapter{Zaključak}
		
		 \texttt{}{Zadatak naše grupe bila je implementacija sustava za naručivanje pregleda u hrvatskom zdravstvu. Kroz našu aplikaciju pacijenti mogu zakazivati termine ovisno o dostupnosti pojedinih doktora/medicinskih radnika. Nakon 14 tjedana rada, naša grupa je ostvarila upotebljivu aplikaciju koja ispunjava sve korisničke zahtjeve koji su bili zadani za nju. }
   
   
          \texttt{}{
          Prvi ciklus izrade projekta zahtjevao je okupljanje tima. Nakon okupljanja, prva podjela tima bila je na frontend i backend domenu. Darijan Gudelj, Branimir Tomeljak i Vilim Ivanković preuzeli su backend domenu, a Luka Slugečić, Bruno Rački i Marin Teskera preuzeli su frontend domenu. Kako je frontend domena bila ograničena poslom, neki članovi frontend domene pomagali su pri izradi backend domene. Nakon izrade prvostupničke aplikacije koja je funkcionirala u minornim zahtjevima, naša grupa je krenula na izradu dokumentacije. Izradili smo sekvencijske dijagrame, obrasce upotrebe, dijagram razreda i model baze podataka. Svi ti modeli olakšali su razumijevanje i izradu novih komponenti u drugom ciklusu rada. Sastanci su se održavali barem jednom na tjednoj bazi, te su članovi NULL grupe ulagali jako puno samostalnog truda i rada između sastanaka.
          }

          \texttt{}{
          Drugi ciklus izrade projekta zahjtevao je da se domene razmijenjuju, tako se većina timova prebacivala po potrebi u drugu domenu. Kako je posla bilo previše, NULL grupa sastajala se dva puta tjedno kako bi uspješno izvršila sve korisničke zahtjeve. U drugom ciklusu radili smo na implementaciji aplikacije koja posjeduje autentifikaciju, nema nepredvidive greške i točno su određene mogućnosti pojedinog korisnika u aplikaciji. Implementirali smo izvješća, SMS poruke, emailove, kalendar itd.. Kako je posao rastao i postajao sve kompleksnije NULL grupa komunicirala je dvjema kanalima, a to su Discord i Whatsapp koji su nam uvelike pomogli u rješavanju problema. Svakodnevno smo komunicirali i izmijenjivali iskustva u izradi ovoga sustava i to sa vjerodostojnošću radili. Nakon završene aplikacije izradili smo i dijagram aktivnosti, stanja i komponenti. 

          Mogućnost izrade ovog projekta svakome članu NULL grupe dala je nova programska i komunikativna iskustva koje nikada nećemo zaboraviti. Svaki član osjetio je koliko je bitno znati raditi u timu i koliko je svačiji rad blagonosan. Kao tim smo se uspjeli organizirati, slušali smo jedni druge i ono što je najbitnije, učili jedni od drugih. 

          Svaki član je zadovoljan postignutim i trudit ćemo se u budućnosti koristiti znanje dobiveno na ovome projektu.
          }
		
		\eject 